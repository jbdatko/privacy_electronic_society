%File: focus_paper.tex
\documentclass[letterpaper]{article}


% Required Packages
\usepackage{aaai}
\usepackage{times}
\usepackage{helvet}
\usepackage{courier}
\usepackage{graphicx}
\usepackage{url}


\begin{document}

%%%%%%%%%%
% Pdfinfo for PDFTEX
% Uncomment and complete the following for metadata if % your paper is
% typeset using PDFTEX
\pdfinfo{
/Title (The Feasibility of Stylometry as a Service)
/Author (Josh Datko and Joseph Heenan)
/Subject (The Feasibility of Stylometry as a Service)
/Keywords (privacy anonymity stylometry )
}
%%%%%%%%%%
% Section Numbers
% Uncomment if you want to use section numbers
% and change the 0 to a 1 or 2
\setcounter{secnumdepth}{1}
%%%%%%%%%%
% Title, Author, and Address Information
\title{The Feasibility of Stylometry as a Service}
\author{Josh Datko \and Joseph Heenan\\ Drexel University\\
CS680: Privacy in the Electronic Society\\
}
%%%%%%%%%%
% Body of Paper Begins \begin{document}
\maketitle

\section*{Introduction}\label{sec:intro}


Maintaining anonymity online is increasingly difficult.  While
networks like Tor \cite{Dingledine04tor:the} will protect an Internet
Protocol (IP) address at the network layer, it does not protect the
content delivered over Tor.  For example, suppose an author desires to
provide controversial political commentary but wants to remain
anonymous.  In this case, she can protect her IP address with Tor and
use end-to-end encryption to protect her message en-route, but she
could be discovered through stylometry.

Stylometry, a form of authorship recognition, has been found to be
extremely effective especially even at
Internet-scale \cite{Narayanan:2012:FIA:2310656.2310687}.  Thus for
those seeking to publish anonymous writings, not only can the delivery
path reveal them, but their content can as well.  Adversarial
stylometry is the active process of manipulating writing to confuse
the identity of the author.  A set of tools, JStylo and
Anonymouth \cite{conf/pet/McDonaldACSG12}, have been produced by the
Privacy, Security, and Automation Lab (PSAL) at Drexel University to
assist authors in anonymize their writing.

Currently the JStylo and Anonymouth integrated open-source project
(JSAN) is available to download and can be installed on a machine that
runs the Java Virtual Machine (JVM).  While JSAN has been successfully
demonstrated to be effective in assisting authors with adversarial
stylometry \cite{journals/tissec/BrennanAG12}, the constraint of
downloading and installing the software on a machine powerful enough
to run machine learning algorithms on data sets, limits their
adoptability.  We propose increase the availability and usability of
JStylo and Anonymouth by offering them as a web service.

Migrating to a web service would provide JSAN with several benefits.
This web service would provide adversarial stylometry capabilities to
embedded devices like tablets or smart-phones which otherwise could
not run the software natively.  Since there would be no installation,
besides a typical web-browser, a web-service would be better suited
for time-critical situations where an author must quickly anonymize
her document.

An example use-case is an author with a time sensitive document that
she needs to release.  Because of the sensitive nature, she wishes to
apply adversarial stylometry to her document.  In this case, a
computer with decent enough performance to natively run machine
learning techniques may not be available.  However, a web service is
nearly ubiquitous.

\subsection{Expanding Stylometric Research}

A stylometric web service also expands research opportunities.  This
service, offered through a modern architecture, would allow
researchers to easily perform experiments.  An easily accessible tool
allows for rapid verification of research results and greater
community engagement.

In addition to convenient access, a web service can provide additional
diversity in stylometric experimentation.  For example, this web
service supports multiple authors and provides a better tool for
real-time, multiple author stylometry than a stand-alone application.

An effective technique in adversarial stylometry is imitation of
another author's writing style.  However, since many are not trained
author imitators, this technique requires significant cognitive effort.
Perhaps a more effective technique, would be to have multiple authors
re-write the target work in their natural style.  While there has been
numerous studies into single author stylometric analysis, multiple
authorship is more difficult.  This web services provides a platform
for multiple authorship research.

\subsection{Challenges in a Web Service}

A privacy enhancing technology that is designed to exist as a web
service faces different challenges than an application.  Attack
vectors open up on the service from active denial-of-service attacks
to passive sniffing.  A privacy web service must take great care to
ensure that service does not compromise its user's privacy.  In
Section \ref{sec:design} we provide our architecture and analysis the
security and privacy.



\section{Design}

\subsection{Anonymouth Architecture Review}

The architecture of Anonymouth is that of a conventional desktop
application.  It uses a layered architecture with essentially two main
components: the frontend Graphical User Interface (GUI) and the
backend, consisting of the Weka machine learning toolkit and Jstylo.
Jystlo is PSAL's stylometric analysis software.   Since Anonymouth was
designed as a monolithic application, there are direct dependencies
between the layers without the use of an adaptation layer.

\section{Security Analysis}

\section{Privacy Analysis}

\section{Implementation Details}

\section{Comparison to Stand-alone Application}

\section{Evaluation}

\section{Conclusion}


\section{Proposal stuff --- will be removed}


\section*{Approach}\label{sec:approach}

We propose to build a stylometric service around JStylo and
Anonymouth.  However, this project is more than just porting JSAN to
the web.  With a migration to a web service from locally running
software, the anonymity attack vector changes significantly.  We
intend to design and evaluate privacy enhancing techniques for
a stylometric web service.

\subsection{Design}

Our design goal is to increase both the anonymity and availability of
JSAN through this service.  In order to increase the anonymity, we
will need to protect the service from traffic analysis, which we
intend to do by running a Tor hidden service.  To improve the
anonymity and effectiveness of JSAN, we intend to incorporate a
third-party re-writer and in the future, a document mixer.  The
document mixer will randomize the target document, potentially with
other target documents, and present a re-writer the mixed document.
The re-writer will be able to use the Anonymouth-like interface to
anonymize the document without being able to infer the author.  Also,
the mixing will help obfuscate the contents of the document as well.
However, for this initial attempt, we will focus on the third-party
anonymization service.

Although in this initial effort we do not intent to conduct usability
testing, we will consider incentive options for re-writers.  Like
other anonymity networks, we believe that anonymity is improved with a
diverse user base.  The challenge is building the user base.  While we
do not intend to conduct user testing for this initial effort, we will
consider various incentive options like actual payment to
gamification.

\subsection{Implementation}

Our initial idea is to host this service on Amazon Web Services
(AWS).  For the user interface, we are planning to implement the front
end in AngularJS.  Since the existing code is Java, migrating to the
Play Framework seems like a natural progression.  The text editor
functionality will be converted to Firepad, an open source
collaborative editor.  While real-time collaboration is not
necessarily needed, we think that an ``Anonymouth'' user, an
Artificial Agent, would assist the re-writer in edits.


\section*{Related Work}\label{sec:related}
This proposal extends the work from Drexel's PSAL, who have made
several contributions in this field.  In
\cite{conf/pet/McDonaldACSG12}, the authors present the Anonymouth
framework for anonymizing writing.
\cite{Afroz:2012:DHF:2310656.2310711} confirmed adversarial stylometry
is very effective at obfuscated authorship, but also presented
techniques to hide the prescience adversarial stylometry.  Lastly,
\cite{journals/tissec/BrennanAG12} presented successful adversarial
stylometry techniques through obfuscation, imitation, and translation
and produced two corpora of texts.

Tor is one of the most successful anonymity services in
operation \cite{Dingledine04tor:the}.  We intend to protect the
anonymity of users by offering JSAN over a Tor hidden service
however, there has been recent research de-anonymizing popular hidden
services \cite{oakland2013-trawling}.  We will provide a discussion on
the design and experience of enabling a hidden service, but we will
focus on this aspect.

\subsection{Novelty}\label{sec:novelty}

There are two significant contributions as a result of this work.  The
first is a stylometric web service.  This will advance the usability
and capabilities of JSAN and provide a framework for more expansive
user testing in the future.  The second contribution is that this is
the first framework for multi-author adversarial stylometry.  While
single author stylometry has been recently studied, multi-author
documents have not received significant attention.

\section*{Evaluation}\label{sec:evaluation}
This effort is intended to produce several artifacts.  The first is an
evaluation of the privacy architecture of a cloud web service.  Here
we intend to describe the challenges and present and evaluate a
privacy-by-architecture system.  Second, we will provide a survey on
various incentive approaches and suggest a possible method.  Lastly,
we will provide a prototype web service and evaluate it.  Specifically
we will compare the web service to the standalone tools and evaluate
it for accuracy and usability.  Additionally, we will experiment with
the third party anonymization by re-writing each other's writing
samples.  We will provide an evaluation on the perceived difficultly
of re-writing one's own work to that in the mixed system.


\section*{Milestones}\label{sec:milestones}

Table \ref{tab:milestones} contains a proposed timeline for completing
this project.  This work will be split among two developers and there
are several modules that can be independently developed.  In order to
re-architect JSAN, we first need to become familiar with the existing
software.  While this software will be running as a service, we do not
intend to release it to the general public in this time-frame.  We
will conduct the evaluation with perhaps a very limited subset of
other students.


\begin{table}
  \centering
  \begin{tabular}{l | c}
    Milestone & Date\\
    \hline

    Document initial design & October 28\\
    Finish porting JSAN to a web service & November 4\\
    Add the Firebase editro & November 11\\
    Add the third party feature & November 18\\
    Conduct evaluation & November 25\\
    Provide Paper and Presentation & December 2

  \end{tabular}
  \caption{Project Milestones}
  \label{tab:milestones}
\end{table}


\subsection*{Future Work}

The scope of this project is designed to completed in one Drexel
quarter, however we have several ideas for continued work.  With a web
service, we can offer real-time adversarial stylometry.  Assume Alice,
who has a blog, wants to anonymize her next blog post.  She would
connect to our web service, submit the URL for her blog and then write
her next post with adversarial stylometry.  This real-time data
crawling at this scale is simply not feasible with stand alone
software.

Now that a third party will be re-writing documents, we feel that it
is important to obfuscate the content from the re-writer.  One idea,
is to mix the document, perhaps with other documents, prior to
re-writing.  The re-writer would be presented with an otherwise
nonsensical document and would focus on re-write each sentence in her
native style.  The combined document, would contain $n$ author styles,
where $n$ is the number of re-writing authors.

Lastly, for long term adversarial projects, we are curious on how often
one has to re-train the model.  We imagine that a web service could
help monitor the change of a stylometric fingerprint over time.


%%%%%%%%%%
% References and End of Paper
\bibliography{service}
\bibliographystyle{aaai}


\end{document}
%%% Local Variables
%%% mode: latex
%%% TeX-master: t
%%% End:
